\tableofcontents

\clearpage

\section{Counting Covers of Elliptic Curves (Orlando Marigliano)}

\emph{Riemann surfaces} are one-dimensional complex manifolds, i.e.
geometric objects that locally look like the complex plane. Their structure
is much more rigid than e.g. the structure of a real surface, hence it is
easier to classify the morphisms between them.

After refreshing the notions of Riemann surfaces, ramified covers, and
elliptic curves, we turn to the problem of computing the number of ramified
covers (i.e. morphisms) of Riemann surfaces into a fixed elliptic curve.

Using general covering theory, we can translate this geometric problem about
surfaces into
a combinatorial one about the symmetric group. The latter may be solved by
algebraic methods, ranging from basic
combinatorics in the symmetric group to the representation and character
theory thereof.

We will see how to apply the technique of generating functions to attack
general counting problems, and how it can help solve our problem.

If time permits, we will see that the generating functions we are interested
in are \emph{quasimodular forms}, a generalization of modular forms.

\section{Ein Geometrischer Beweis des Satzes von Hurewicz
(Thorben Kastenholz)}

Der Satz von Hurewicz ist ein wohlbekannter Satz aus der algebraischen
Topologie, der tagtäglich Anwendung findet.

\begin{satz}
Sei $(X,x_0)$ ein $(n-1)$-zusammenhängender punktierter Raum, wobei $n\geq 1$
ist. Dann gilt für $n=1$ $(\pi_1(X,x_0))^{\mathrm{ab}} \cong H_1(X)$ und falls
$n \geq 2$ gilt $\pi_k(X,x_0) \cong H_k(X)$ für alle $k\leq n$. All dies
Isomorphismen sind natürlich.
\end{satz}

Die Theorie der $p$-Stratifolds erlaubt es diesen Satz mit sehr geometrischen
Mitteln zu beweisen. Das Entscheidende hier ist, dass $p$-Stratifolds eine
Homologietheorie bilden, ähnlich wie Manigfaltigkeiten bis auf Bordismen eine
Homologietheorie bilden. Diese ist für alle Räume natürlich isomorph zu
singulärer Homologie, jedoch lassen sich $p$-Stratifolds in vielen Fällen wie
Mannigfaltigkeiten händeln und erlauben so sehr geometrische Manipulationen
die letzten Endes diesen Satz beweisen werden.

Mit ähnlichen Mitteln lässt sich noch das folgende Resultat beweisen:
\begin{satz}
Sei $n \geq 3$ und $(X,x_0)$ ein $(n-1)$-zusammenhängender punktierter Raum
dann ist die folgende Sequenz exakt und natürlich:
\[
\pi_n(X,x_0)/2\pi_n(X,x_o) \xrightarrow{- \circ h_n} \pi_{n+1}(X,x_0)
\xrightarrow{\hspace{1ex} H_n \hspace{1ex}} H_{n+1}(X) \xrightarrow{} 0
\]
Hier stehen $H_n$ für die entsprechende Hurewiczabbildung und $h_n$ für die
$(n-2)$-te Einhängung der Hopfabbildung $h_2 \colon S^3 \to S^2$. Für $n = 2$
gilt immerhin, dass $H_2 \colon \pi_{n+1}(X,x_0) \to H_{n+1}(X)$ surjektiv ist.
\end{satz}

\section{Quillen's Plus-Construction and Algebraic $K$-Theory (Leon Hendrian)}

\

\section{Loop Objects in Pointed Derivators (Aras Ergus)}

Derivators provide an abstract framework for homotopy
theory. In particular, many statements from (classical) homotopy theory
can be formulated and proven for certain kinds of derivators. My thesis
is about one such statement, namely a ``derivator version'' of the fact that
the loop spaces have a canonical group object structure in the homotopy
category of topological spaces.

\section{Spectral Theorem for Unbounded Operators (Thomas Bodendorfer)}

\

\section{Singular Integrals (Nikolay Barashkov)}

\ 
