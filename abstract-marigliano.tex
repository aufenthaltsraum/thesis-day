\section{Counting Covers of Elliptic Curves (Orlando Marigliano)}

\emph{Riemann surfaces} are one-dimensional complex manifolds, i.e.
geometric objects that locally look like the complex plane. Their structure
is much more rigid than e.g. the structure of a real surface, hence it is
easier to classify the morphisms between them.

After refreshing the notions of Riemann surfaces, ramified covers, and
elliptic curves, we turn to the problem of computing the number of ramified
covers (i.e. morphisms) of Riemann surfaces into a fixed elliptic curve.

Using general covering theory, we can translate this geometric problem about
surfaces into
a combinatorial one about the symmetric group. The latter may be solved by
algebraic methods, ranging from basic
combinatorics in the symmetric group to the representation and character
theory thereof.

We will see how to apply the technique of generating functions to attack
general counting problems, and how it can help solve our problem.

If time permits, we will see that the generating functions we are interested
in are \emph{quasimodular forms}, a generalization of modular forms.
