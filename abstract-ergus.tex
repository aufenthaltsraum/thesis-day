\section{Loop Objects in Pointed Derivators (Aras Ergus)}

\textbf{TL;DR:} \emph{Derivators provide an abstract framework for homotopy
theory. In particular, many statements from (classical) homotopy theory
can be formulated and proven for certain kinds of derivators. My thesis
is about one such statement, namely a ``derivator version'' of the fact that
the loop spaces have a canonical group object structure in the homotopy
category of pointed topological spaces.}

In some fields of mathematics, especially in homotopy theory and
homological algebra, one wants to see certain kinds of morphisms
(week equivalences or homotopy equivalences of topological spaces
in the first case, quasi-isomorphisms of chain complexes in the second
case) as isomorphisms. Given a category $\mathbf{C}$ and a class $W$
of ``weak equivalences'' which one wants to see as isomorphisms,
one can, in certain cases which include the ones above, construct the
``homotopy category'' $W^{-1}\mathbf{C}$ equipped with a functor
$Q \colon \mathbf{C} \to W^{-1}\mathbf{C}$ such that $Q(f)$ is
an isomorphism whenever $f \in W$ and $(W^{-1}\mathbf{C}, Q)$ is
universal with this property.

A question one could ask in this situation is whether $W^{-1}\mathbf{C}$
has all (co)limits if $\mathbf{C}$ does. The answer to this
question turns out to be negative in most non-trivial cases.

In order to find a replacement for (co)limits in $W^{-1}\mathbf{C}$
one considers the following alternative description of (co)limits:
Given a small category $A$ and a category $\mathbf{C}$ which has all limits
and colimits, there are adjunctions
\[
\Delta \colon \mathbf{C} \leftrightarrows \mathbf{C}^A \colon \lim
\quad \text{and} \quad
\operatorname{colim} \colon \mathbf{C}^A \leftrightarrows
\mathbf{C} \colon \Delta,
\]
where $\mathbf{C}^A$ is the category of functors from $A$ to $\mathbf{C}$
with natural transformations as morphisms and
$\Delta \colon \mathbf{C} \to \mathbf{C}^A$ is the ``constant diagram
functor'' which maps each object $c$ of $\mathbf{C}$ to the functor which is
contantly $c$ on objects of $A$ and constantly $\operatorname{id}_c$ on
morphisms in $A$.

Now the idea is to replace $(W^{-1}\mathbf{C})^A$ by
$W_A^{-1}(\mathbf{C}^A)$, where $W_A$ is the class of morphisms in
$\mathbf{C}^A$ (i.\,e.\ natural transformations) whose all components lie in
$W$. This distinction between ``diagrams in the homotopy category'' and
''diagrams up to pointwise weak equivalences'' turns out to be very important
and one in fact does have adjunctions
\[
\Delta^{\mathrm{h}} \colon W^{-1}\mathbf{C} \leftrightarrows
W_A^{-1}(\mathbf{C}^A) \colon \operatorname{holim}
\quad \text{and} \quad
\operatorname{hocolim} \colon W_A^{-1}(\mathbf{C}^A) \leftrightarrows
W^{-1}\mathbf{C} \colon \Delta^{\mathrm{h}}
\]
if $\mathbf{C}$ has all limits and colimits, where
$\Delta^{\mathrm{h}} \colon W^{-1}\mathbf{C} \to W_A^{-1}(\mathbf{C}^A)$ is
again a ``constant diagram functor'' similar to the one above.

From this point on many statements in homotopy theory and homological
algebra can be shown purely formally using such adjunctions. The concept of a
\emph{derivator} provides an axiomatization of this phenomenon
which describes the totality of ``categories of coherent diagrams'' and
certain adjunctions between those.

For example, for a large class of derivators one can define a loop
functor which corresponds to the loop space functor in the case of the
homotopy theory of pointed topological spaces and to the shift functor
in homological algebra. A well-known statement in homotopy theory
is that loop spaces have a group object structure in the homotopy category
(which is given by concatenation and inversion of loops). One can indeed
show a ``derivator version'' of this statement, which is the topic
of my thesis.
